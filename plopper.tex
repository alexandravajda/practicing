\documentclass[]{article}
\usepackage{lmodern}
\usepackage{amssymb,amsmath}
\usepackage{ifxetex,ifluatex}
\usepackage{fixltx2e} % provides \textsubscript
\ifnum 0\ifxetex 1\fi\ifluatex 1\fi=0 % if pdftex
  \usepackage[T1]{fontenc}
  \usepackage[utf8]{inputenc}
\else % if luatex or xelatex
  \ifxetex
    \usepackage{mathspec}
  \else
    \usepackage{fontspec}
  \fi
  \defaultfontfeatures{Ligatures=TeX,Scale=MatchLowercase}
\fi
% use upquote if available, for straight quotes in verbatim environments
\IfFileExists{upquote.sty}{\usepackage{upquote}}{}
% use microtype if available
\IfFileExists{microtype.sty}{%
\usepackage{microtype}
\UseMicrotypeSet[protrusion]{basicmath} % disable protrusion for tt fonts
}{}
\usepackage[margin=1in]{geometry}
\usepackage{hyperref}
\hypersetup{unicode=true,
            pdftitle={Plopper},
            pdfauthor={Alexandra Vajda},
            pdfborder={0 0 0},
            breaklinks=true}
\urlstyle{same}  % don't use monospace font for urls
\usepackage{color}
\usepackage{fancyvrb}
\newcommand{\VerbBar}{|}
\newcommand{\VERB}{\Verb[commandchars=\\\{\}]}
\DefineVerbatimEnvironment{Highlighting}{Verbatim}{commandchars=\\\{\}}
% Add ',fontsize=\small' for more characters per line
\usepackage{framed}
\definecolor{shadecolor}{RGB}{248,248,248}
\newenvironment{Shaded}{\begin{snugshade}}{\end{snugshade}}
\newcommand{\KeywordTok}[1]{\textcolor[rgb]{0.13,0.29,0.53}{\textbf{#1}}}
\newcommand{\DataTypeTok}[1]{\textcolor[rgb]{0.13,0.29,0.53}{#1}}
\newcommand{\DecValTok}[1]{\textcolor[rgb]{0.00,0.00,0.81}{#1}}
\newcommand{\BaseNTok}[1]{\textcolor[rgb]{0.00,0.00,0.81}{#1}}
\newcommand{\FloatTok}[1]{\textcolor[rgb]{0.00,0.00,0.81}{#1}}
\newcommand{\ConstantTok}[1]{\textcolor[rgb]{0.00,0.00,0.00}{#1}}
\newcommand{\CharTok}[1]{\textcolor[rgb]{0.31,0.60,0.02}{#1}}
\newcommand{\SpecialCharTok}[1]{\textcolor[rgb]{0.00,0.00,0.00}{#1}}
\newcommand{\StringTok}[1]{\textcolor[rgb]{0.31,0.60,0.02}{#1}}
\newcommand{\VerbatimStringTok}[1]{\textcolor[rgb]{0.31,0.60,0.02}{#1}}
\newcommand{\SpecialStringTok}[1]{\textcolor[rgb]{0.31,0.60,0.02}{#1}}
\newcommand{\ImportTok}[1]{#1}
\newcommand{\CommentTok}[1]{\textcolor[rgb]{0.56,0.35,0.01}{\textit{#1}}}
\newcommand{\DocumentationTok}[1]{\textcolor[rgb]{0.56,0.35,0.01}{\textbf{\textit{#1}}}}
\newcommand{\AnnotationTok}[1]{\textcolor[rgb]{0.56,0.35,0.01}{\textbf{\textit{#1}}}}
\newcommand{\CommentVarTok}[1]{\textcolor[rgb]{0.56,0.35,0.01}{\textbf{\textit{#1}}}}
\newcommand{\OtherTok}[1]{\textcolor[rgb]{0.56,0.35,0.01}{#1}}
\newcommand{\FunctionTok}[1]{\textcolor[rgb]{0.00,0.00,0.00}{#1}}
\newcommand{\VariableTok}[1]{\textcolor[rgb]{0.00,0.00,0.00}{#1}}
\newcommand{\ControlFlowTok}[1]{\textcolor[rgb]{0.13,0.29,0.53}{\textbf{#1}}}
\newcommand{\OperatorTok}[1]{\textcolor[rgb]{0.81,0.36,0.00}{\textbf{#1}}}
\newcommand{\BuiltInTok}[1]{#1}
\newcommand{\ExtensionTok}[1]{#1}
\newcommand{\PreprocessorTok}[1]{\textcolor[rgb]{0.56,0.35,0.01}{\textit{#1}}}
\newcommand{\AttributeTok}[1]{\textcolor[rgb]{0.77,0.63,0.00}{#1}}
\newcommand{\RegionMarkerTok}[1]{#1}
\newcommand{\InformationTok}[1]{\textcolor[rgb]{0.56,0.35,0.01}{\textbf{\textit{#1}}}}
\newcommand{\WarningTok}[1]{\textcolor[rgb]{0.56,0.35,0.01}{\textbf{\textit{#1}}}}
\newcommand{\AlertTok}[1]{\textcolor[rgb]{0.94,0.16,0.16}{#1}}
\newcommand{\ErrorTok}[1]{\textcolor[rgb]{0.64,0.00,0.00}{\textbf{#1}}}
\newcommand{\NormalTok}[1]{#1}
\usepackage{graphicx,grffile}
\makeatletter
\def\maxwidth{\ifdim\Gin@nat@width>\linewidth\linewidth\else\Gin@nat@width\fi}
\def\maxheight{\ifdim\Gin@nat@height>\textheight\textheight\else\Gin@nat@height\fi}
\makeatother
% Scale images if necessary, so that they will not overflow the page
% margins by default, and it is still possible to overwrite the defaults
% using explicit options in \includegraphics[width, height, ...]{}
\setkeys{Gin}{width=\maxwidth,height=\maxheight,keepaspectratio}
\IfFileExists{parskip.sty}{%
\usepackage{parskip}
}{% else
\setlength{\parindent}{0pt}
\setlength{\parskip}{6pt plus 2pt minus 1pt}
}
\setlength{\emergencystretch}{3em}  % prevent overfull lines
\providecommand{\tightlist}{%
  \setlength{\itemsep}{0pt}\setlength{\parskip}{0pt}}
\setcounter{secnumdepth}{0}
% Redefines (sub)paragraphs to behave more like sections
\ifx\paragraph\undefined\else
\let\oldparagraph\paragraph
\renewcommand{\paragraph}[1]{\oldparagraph{#1}\mbox{}}
\fi
\ifx\subparagraph\undefined\else
\let\oldsubparagraph\subparagraph
\renewcommand{\subparagraph}[1]{\oldsubparagraph{#1}\mbox{}}
\fi

%%% Use protect on footnotes to avoid problems with footnotes in titles
\let\rmarkdownfootnote\footnote%
\def\footnote{\protect\rmarkdownfootnote}

%%% Change title format to be more compact
\usepackage{titling}

% Create subtitle command for use in maketitle
\providecommand{\subtitle}[1]{
  \posttitle{
    \begin{center}\large#1\end{center}
    }
}

\setlength{\droptitle}{-2em}

  \title{Plopper}
    \pretitle{\vspace{\droptitle}\centering\huge}
  \posttitle{\par}
    \author{Alexandra Vajda}
    \preauthor{\centering\large\emph}
  \postauthor{\par}
      \predate{\centering\large\emph}
  \postdate{\par}
    \date{22/05/2019}


\begin{document}
\maketitle

\begin{Shaded}
\begin{Highlighting}[]
\NormalTok{mydata <-}\StringTok{ }\KeywordTok{read.csv}\NormalTok{(}\DataTypeTok{file =} \StringTok{"3.1.txt"}\NormalTok{, }\DataTypeTok{sep =} \StringTok{","}\NormalTok{, }\DataTypeTok{header =} \OtherTok{TRUE}\NormalTok{)}
\KeywordTok{dim}\NormalTok{(mydata)}
\end{Highlighting}
\end{Shaded}

\begin{verbatim}
## [1] 33988     5
\end{verbatim}

\begin{Shaded}
\begin{Highlighting}[]
\KeywordTok{str}\NormalTok{(mydata)}
\end{Highlighting}
\end{Shaded}

\begin{verbatim}
## 'data.frame':    33988 obs. of  5 variables:
##  $ caseid  : int  339 340 345 346 352 353 354 361 362 363 ...
##  $ score   : int  49 18 46 43 17 29 15 19 45 12 ...
##  $ cohort90: int  -6 -6 -6 -6 -6 -6 -6 -6 -6 -6 ...
##  $ female  : int  0 0 0 0 0 0 0 0 0 0 ...
##  $ sclass  : int  2 3 4 3 3 2 3 2 3 1 ...
\end{verbatim}

\begin{Shaded}
\begin{Highlighting}[]
\NormalTok{mydata [}\DecValTok{1}\OperatorTok{:}\DecValTok{20}\NormalTok{, ]}
\end{Highlighting}
\end{Shaded}

\begin{verbatim}
##    caseid score cohort90 female sclass
## 1     339    49       -6      0      2
## 2     340    18       -6      0      3
## 3     345    46       -6      0      4
## 4     346    43       -6      0      3
## 5     352    17       -6      0      3
## 6     353    29       -6      0      2
## 7     354    15       -6      0      3
## 8     361    19       -6      0      2
## 9     362    45       -6      0      3
## 10    363    12       -6      0      1
## 11   6824     0       -4      0      1
## 12   6826     0       -4      0      3
## 13   6827    20       -4      0      2
## 14   6828    32       -4      0      1
## 15   6829     0       -4      0      2
## 16   6834    24       -4      0      3
## 17   6836    23       -4      0      2
## 18  13206     7       -2      0      3
## 19  13209    38       -2      0      3
## 20  13215    46       -2      0      1
\end{verbatim}

\begin{Shaded}
\begin{Highlighting}[]
\KeywordTok{hist}\NormalTok{(mydata}\OperatorTok{$}\NormalTok{score, }\DataTypeTok{xlim =} \KeywordTok{c}\NormalTok{(}\DecValTok{0}\NormalTok{,}\DecValTok{80}\NormalTok{))}
\end{Highlighting}
\end{Shaded}

\includegraphics{plopper_files/figure-latex/unnamed-chunk-3-1.pdf}

\begin{Shaded}
\begin{Highlighting}[]
\KeywordTok{summary}\NormalTok{(mydata)}
\end{Highlighting}
\end{Shaded}

\begin{verbatim}
##      caseid          score          cohort90           female      
##  Min.   :    1   Min.   : 0.00   Min.   :-6.0000   Min.   :0.0000  
##  1st Qu.: 8532   1st Qu.:19.00   1st Qu.:-4.0000   1st Qu.:0.0000  
##  Median :17318   Median :33.00   Median :-2.0000   Median :1.0000  
##  Mean   :18466   Mean   :31.09   Mean   : 0.2767   Mean   :0.5276  
##  3rd Qu.:29428   3rd Qu.:45.00   3rd Qu.: 6.0000   3rd Qu.:1.0000  
##  Max.   :38192   Max.   :75.00   Max.   : 8.0000   Max.   :1.0000  
##      sclass     
##  Min.   :1.000  
##  1st Qu.:1.000  
##  Median :2.000  
##  Mean   :2.147  
##  3rd Qu.:3.000  
##  Max.   :4.000
\end{verbatim}

\begin{Shaded}
\begin{Highlighting}[]
\KeywordTok{summary}\NormalTok{(mydata}\OperatorTok{$}\NormalTok{score)}
\end{Highlighting}
\end{Shaded}

\begin{verbatim}
##    Min. 1st Qu.  Median    Mean 3rd Qu.    Max. 
##    0.00   19.00   33.00   31.09   45.00   75.00
\end{verbatim}

\begin{Shaded}
\begin{Highlighting}[]
\KeywordTok{sd}\NormalTok{(mydata}\OperatorTok{$}\NormalTok{score)}
\end{Highlighting}
\end{Shaded}

\begin{verbatim}
## [1] 17.31437
\end{verbatim}

\begin{Shaded}
\begin{Highlighting}[]
\KeywordTok{hist}\NormalTok{(mydata}\OperatorTok{$}\NormalTok{cohort90, }\DataTypeTok{xlim =} \KeywordTok{c}\NormalTok{(}\OperatorTok{-}\DecValTok{6}\NormalTok{, }\DecValTok{8}\NormalTok{))}
\end{Highlighting}
\end{Shaded}

\includegraphics{plopper_files/figure-latex/unnamed-chunk-5-1.pdf}

\begin{Shaded}
\begin{Highlighting}[]
\NormalTok{mydata <-}\StringTok{ }\KeywordTok{read.csv}\NormalTok{(}\StringTok{"3.1.txt"}\NormalTok{, }\DataTypeTok{sep =} \StringTok{","}\NormalTok{, }\DataTypeTok{header =} \OtherTok{TRUE}\NormalTok{)}

\NormalTok{mytable <-}\StringTok{ }\KeywordTok{table}\NormalTok{(mydata}\OperatorTok{$}\NormalTok{cohort90)}
\NormalTok{mytable}
\end{Highlighting}
\end{Shaded}

\begin{verbatim}
## 
##   -6   -4   -2    0    6    8 
## 6478 6325 5245 4371 4244 7325
\end{verbatim}

\begin{Shaded}
\begin{Highlighting}[]
\KeywordTok{prop.table}\NormalTok{(mytable)}
\end{Highlighting}
\end{Shaded}

\begin{verbatim}
## 
##        -6        -4        -2         0         6         8 
## 0.1905967 0.1860951 0.1543192 0.1286042 0.1248676 0.2155172
\end{verbatim}

\begin{Shaded}
\begin{Highlighting}[]
\NormalTok{mytablecomb <-}\StringTok{ }\KeywordTok{cbind}\NormalTok{(mytable, }\KeywordTok{prop.table}\NormalTok{(mytable), }\KeywordTok{cumsum}\NormalTok{(}\KeywordTok{prop.table}\NormalTok{(mytable)))}
\NormalTok{mytablecomb}
\end{Highlighting}
\end{Shaded}

\begin{verbatim}
##    mytable                    
## -6    6478 0.1905967 0.1905967
## -4    6325 0.1860951 0.3766918
## -2    5245 0.1543192 0.5310109
## 0     4371 0.1286042 0.6596152
## 6     4244 0.1248676 0.7844828
## 8     7325 0.2155172 1.0000000
\end{verbatim}

\begin{Shaded}
\begin{Highlighting}[]
\KeywordTok{colnames}\NormalTok{(mytablecomb) <-}\StringTok{ }\KeywordTok{c}\NormalTok{(}\StringTok{"Freq"}\NormalTok{, }\StringTok{"Perc"}\NormalTok{, }\StringTok{"Cum"}\NormalTok{)}
\NormalTok{mytablecomb}
\end{Highlighting}
\end{Shaded}

\begin{verbatim}
##    Freq      Perc       Cum
## -6 6478 0.1905967 0.1905967
## -4 6325 0.1860951 0.3766918
## -2 5245 0.1543192 0.5310109
## 0  4371 0.1286042 0.6596152
## 6  4244 0.1248676 0.7844828
## 8  7325 0.2155172 1.0000000
\end{verbatim}

\begin{Shaded}
\begin{Highlighting}[]
\NormalTok{mytablecomb}
\end{Highlighting}
\end{Shaded}

\begin{verbatim}
##    Freq      Perc       Cum
## -6 6478 0.1905967 0.1905967
## -4 6325 0.1860951 0.3766918
## -2 5245 0.1543192 0.5310109
## 0  4371 0.1286042 0.6596152
## 6  4244 0.1248676 0.7844828
## 8  7325 0.2155172 1.0000000
\end{verbatim}

\begin{Shaded}
\begin{Highlighting}[]
\KeywordTok{plot}\NormalTok{(mydata}\OperatorTok{$}\NormalTok{cohort90, mydata}\OperatorTok{$}\NormalTok{score, }\DataTypeTok{ylim =} \KeywordTok{c}\NormalTok{(}\DecValTok{0}\NormalTok{,}\DecValTok{80}\NormalTok{))}
\end{Highlighting}
\end{Shaded}

\includegraphics{plopper_files/figure-latex/unnamed-chunk-11-1.pdf}
\#Tapply command applies a function to each group of values given by the
levels of the specified factor. \#The first line calculates the lenght
of score for each value of cohort90, providing the number of
observations for each level of cohort90. \#The second line calculates
the mean value of score separately for each value of cohort90. \#The
third line calculates the standard deviation of score for each cohort.

\begin{Shaded}
\begin{Highlighting}[]
\NormalTok{l <-}\StringTok{ }\KeywordTok{tapply}\NormalTok{(mydata}\OperatorTok{$}\NormalTok{score, }\KeywordTok{factor}\NormalTok{(mydata}\OperatorTok{$}\NormalTok{cohort90), length)}
\NormalTok{m <-}\StringTok{ }\KeywordTok{tapply}\NormalTok{(mydata}\OperatorTok{$}\NormalTok{score, }\KeywordTok{factor}\NormalTok{(mydata}\OperatorTok{$}\NormalTok{cohort90), mean)}
\NormalTok{s <-}\StringTok{ }\KeywordTok{tapply}\NormalTok{(mydata}\OperatorTok{$}\NormalTok{score, }\KeywordTok{factor}\NormalTok{(mydata}\OperatorTok{$}\NormalTok{cohort90), sd)}
\NormalTok{tableScore <-}\StringTok{ }\KeywordTok{cbind}\NormalTok{(}\StringTok{"Freq"}\NormalTok{ =}\StringTok{ }\NormalTok{l, }\StringTok{"mean(score)"}\NormalTok{=}\StringTok{ }\NormalTok{m, }\StringTok{"sd(score)"}\NormalTok{ =}\StringTok{ }\NormalTok{s)}
\NormalTok{tableScore}
\end{Highlighting}
\end{Shaded}

\begin{verbatim}
##    Freq mean(score) sd(score)
## -6 6478    23.65545  18.07995
## -4 6325    24.77265  17.37533
## -2 5245    28.52450  15.93629
## 0  4371    29.10043  15.76355
## 6  4244    39.43473  13.55147
## 8  7325    41.33065  13.00926
\end{verbatim}

\begin{Shaded}
\begin{Highlighting}[]
\KeywordTok{cor}\NormalTok{(mydata}\OperatorTok{$}\NormalTok{score, mydata}\OperatorTok{$}\NormalTok{cohort90)}
\end{Highlighting}
\end{Shaded}

\begin{verbatim}
## [1] 0.4088625
\end{verbatim}

\section{Linear regression model}\label{linear-regression-model}

\begin{Shaded}
\begin{Highlighting}[]
\NormalTok{fit <-}\StringTok{ }\KeywordTok{lm}\NormalTok{(score }\OperatorTok{~}\StringTok{ }\NormalTok{cohort90, }\DataTypeTok{data =}\NormalTok{ mydata)}
\KeywordTok{summary}\NormalTok{(fit)}
\end{Highlighting}
\end{Shaded}

\begin{verbatim}
## 
## Call:
## lm(formula = score ~ cohort90, data = mydata)
## 
## Residuals:
##    Min     1Q Median     3Q    Max 
## -41.31 -11.73   0.56  12.20  46.20 
## 
## Coefficients:
##             Estimate Std. Error t value Pr(>|t|)    
## (Intercept) 30.72873    0.08582  358.04   <2e-16 ***
## cohort90     1.32214    0.01601   82.59   <2e-16 ***
## ---
## Signif. codes:  0 '***' 0.001 '**' 0.01 '*' 0.05 '.' 0.1 ' ' 1
## 
## Residual standard error: 15.8 on 33986 degrees of freedom
## Multiple R-squared:  0.1672, Adjusted R-squared:  0.1671 
## F-statistic:  6822 on 1 and 33986 DF,  p-value: < 2.2e-16
\end{verbatim}

\section{Adding the linear plot}\label{adding-the-linear-plot}

\begin{Shaded}
\begin{Highlighting}[]
\NormalTok{predscore <-}\StringTok{ }\KeywordTok{predict}\NormalTok{(fit)}
\KeywordTok{plot}\NormalTok{(mydata}\OperatorTok{$}\NormalTok{cohort90, predscore, }\DataTypeTok{type =} \StringTok{"l"}\NormalTok{)}
\end{Highlighting}
\end{Shaded}

\includegraphics{plopper_files/figure-latex/unnamed-chunk-15-1.pdf}
\#The previous method was not efficient computationally. We simply need
to run the command for a subset of data that gives us the combinations
of the two variables used to plot the graph. First, we need to make a
matrix with the two objects and then reduce it with the unique command.

\begin{Shaded}
\begin{Highlighting}[]
\NormalTok{uniquedata <-}\StringTok{ }\KeywordTok{cbind}\NormalTok{(}\DataTypeTok{cohort90 =}\NormalTok{ mydata}\OperatorTok{$}\NormalTok{cohort90, }\DataTypeTok{predscore =}\NormalTok{ predscore)}
\NormalTok{uniquedata <-}\StringTok{ }\KeywordTok{unique}\NormalTok{(uniquedata)}
\NormalTok{uniquedata}
\end{Highlighting}
\end{Shaded}

\begin{verbatim}
##    cohort90 predscore
## 1        -6  22.79586
## 11       -4  25.44015
## 18       -2  28.08444
## 23        0  30.72873
## 27        6  38.66159
## 31        8  41.30588
\end{verbatim}

\begin{Shaded}
\begin{Highlighting}[]
\KeywordTok{plot}\NormalTok{(uniquedata[, }\DecValTok{1}\NormalTok{], uniquedata[,}\DecValTok{2}\NormalTok{], }\DataTypeTok{xlab =} \StringTok{"cohort"}\NormalTok{, }\DataTypeTok{ylab =} \StringTok{"predscore"}\NormalTok{, }\DataTypeTok{type =} \StringTok{"l"}\NormalTok{)}
\end{Highlighting}
\end{Shaded}

\includegraphics{plopper_files/figure-latex/unnamed-chunk-17-1.pdf}
\#\#Model checking: checking assumptions. The score value has a high
proportion of zeros. We have to check the distribution of the residuals.
We will look at two plot: 1. standardised residuals vs.~normal scores
(normal plot), and 2. standardised residuals vs fixed part prediction.


\end{document}
